\documentclass[a4paper,12pt]{article}
\usepackage[MeX]{polski}
\usepackage[utf8]{inputenc}

\title{Wzory matematyczne}
\author{Łukasz Szymborski}
\date{24 października 2017}
\begin{document}
\maketitle

\begin{abstract}
Dokument zawiera wzory ćwiczeniowe do zrozumienia jak robić formuły matematyczne w LaTeXie. Reszta zadań: https://github.com/lukzmu/university-pu 
\end{abstract}

\section{Ćwiczenie 3 - wzory matematyczne}

\begin{equation}
%
\lim_{n \to \infty}
\sum_{k=1}^n \frac{1}{k^2}
= \frac{\pi^2}{6} 
%
\end{equation}
\begin{equation}
%
\prod_{i=2}^{n=i^2}
= \frac{\lim^{n \to 4}(1+ \frac{1}{n})^n}{\sum k(\frac{1}{n})}
%
\end{equation}

Łatwo równanie (1) jest doprowadzić do (2)

\begin{equation}
%
\int_{2}^{\infty} \frac{1}{\log_{2}x}dx
= \frac{1}{x}sinx = 1 - cos^2(x)
%
\end{equation}
\begin{equation}
%
\left[ \begin{array}{cccc}
a_{11} & a_{12} & \ldots & a_{1K} \\
a_{21} & a_{22} & \ldots & a_{2K} \\
\vdots & \vdots & \ddots & \vdots \\
a_{K1} & a_{K2} & \ldots & a_{KK}
\end{array} \right] * 
\left[ \begin{array}{c}
x_{1} \\
x_{2} \\
\vdots \\
x_{K} 
\end{array} \right] =
\left[ \begin{array}{c}
b_{1} \\
b_{2} \\
\vdots \\
b_{K}
\end{array} \right]
%
\end{equation}
\begin{equation}
%
(a_{1}=a_{1}(x)) \wedge (a_{2}=a_{2}(x)) \wedge \ldots \wedge (a_{k}=a_{k}(x)) \twoheadrightarrow (d=d(u))
%
\end{equation}

\end{document}