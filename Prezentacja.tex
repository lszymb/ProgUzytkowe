\documentclass{beamer}
\usepackage{amsfonts}
\usepackage[MeX]{polski}
\usepackage[utf8]{inputenc}
\usepackage{graphicx}

%opening
\title{Prezentacja}
\author{Łukasz Szymborski}
\date{07 listopada 2017}
\begin{document}

\frame{\titlepage}
\begin{frame}
\frametitle{Spis treści}
\tableofcontents
\end{frame}
\section{NVidia}
\begin{frame}{Definicja}
Nvidia Corporation – amerykańska firma komputerowa będąca jednym z największych na świecie producentów procesorów graficznych i innych układów scalonych przeznaczonych na rynek komputerowy.
\begin{figure}
\centering
\includegraphics[scale=0.2]{nvidia.jpg}
\end{figure}
\end{frame}
\begin{frame}{Informacje}
\begin{itemize}
\item<1-2> Nvidia jest także głównym dostawcą (pod względem udziału w rynku) kart graficznych dla komputerów osobistych ze swoją standardową serią GeForce. Firma produkuje także konsole (Nvidia Shield) oparte na Androidzie.
\pause
\item<-2> Logo firmy to zielony prostokąt, na który częściowo zachodzi spirala.
\end{itemize}
\end{frame}
\begin{frame}{Układy graficzne}
Przykładowe układy graficzne:
\begin{itemize}
\item<1-5> GeForce - powszechne układu domowe
\item<2-5> Quadro - układy dla architektów
\item<3-5> Tesla - do obliczeń naukowych
\item<4-5> XGPU - w konsolach Xbox
\item<5> RSX - w konsolach PS3 przy współpracy z Sony
\end{itemize}
\end{frame}
\end{document}